\chapter{Procedimientos}    

\section{Verificaci\'{o}n del funcionamiento de los tubos fotomultiplicadores} 

 El primer paso en la implementaci\'{o}n de C2 y C3 fue la revisi\'{o}n de cada una  
de las c\'{e}lulas fotoel\'{e}ctricas para la verificaci\'{o}n de su correcto  
funcionamiento.\\
%
 Para cada uno de los 110 tubos fotomultiplicadores de C2 y los 100 de C3 se  
vari\'{o} el voltaje de alimentaci\'{o}n hasta obtener una se\~{n}al promedio de 20mV  
originada por emisi\'{o}n t\'{e}rmica de electrones. Este procedimiento permiti\'{o} la  
identificaci\'{o}n de los tubos fotomultiplicadores defectuosos y de los que no ten\'{\i}an  
una ganancia suficiente para obtener la eficiencia deseada en los detectores. Adem\'{a}s  
proporcion\'{o} un punto de partida para encontrar el voltaje adecuado para la ganancia  
que se necesita para tener una lectura clara de las se\~{n}ales en los ADC�s.\\
%
 El resultado de este procedimiento dio como resultado el hallazgo de XX tubos  
defectuosos, X en C2 y Y en C3. La mayor\'{\i}a de estos mostraban fracturas en la zona  
donde se encontraban los conectores de los diferentes d\'{\i}nodos del fototubo, en  
algunos casos muy peque\~{n}as, pero suficientes para causar una p\'{e}rdida del  
vac\'{\i}o en los tubos y oxidar los c\'{a}todos. De esta manera, al oxidarse, los  
c\'{a}todos pierden la capacidad de emitir electrones t\'{e}rmicos o fotoelectrones en  
presencia de luz Cerenkov.\\

 Otro de los defectos hallados en varios tubos fotomultiplicadores de C3 fue el  
relacionado con la forma del pulso el\'{e}ctrico obtenido de ellos. En varios casos la  
se\~{n}al era muy irregular mostrando zonas de repeticiones continuas de pulsos  
secundarios y un l\'{\i}mite final que ca\'{\i}a en un tiempo largo comparado con las  
se\~{n}ales m\'{a}s limpias de los tubos de comportamiento normal ( ver figura  
se\~{n}ales ).\\

 Una posible explicaci\'{o}n al comportamiento de estos tubos es que hayan sido  
contaminados con Helio. Es posible que Helio, el radiador de C3, se haya difundido a  
trav\'{e}s de las ventanas de estos fototubos y, en consecuencia ser responsable de  
emisi\'{o}n de electrones por ionizaci\'{o}n causando subsecuentes pulsos electricos  
acompa\~{n}ando a la se\~{n}al original.\\

 Los tubos defectuosos fueron reemplazados con tubos provenientes de detectores de  
experimentos anteriores, los que, a su vez, fueron probados siguiendo el mismo  
procedimiento y se encuentran listos en necesidad de reemplazo.\\

\section{Revisi\'{o}n del sistema de alto voltaje} 
El sistema de alto voltaje para C2 y C3 es el 1440 de LeCroy que consiste de un  
bastidor con capacidad para 16 tarjetas de 16 canales de alto voltaje cada una. Un  
m\'{o}dulo controlador se encarga de comunicarse con la red de computadores y  
controlar cada una de las tarjetas de alto voltaje. El sistema consta, adem\'{a}s, de dos  
m\'{o}dulos que sirven de fuente de poder para todo el sistema.\\

La revisi\'{o}n de este sistema consisti\'{o}, en primer lugar, de la inserci\'{o}n en la  
base de datos de los valores preliminares de alto voltaje para todos los canales de C2 y  
C3. Cada entrada incluye el nombre y n\'{u}mero del canal, el n\'{u}mero de bastidor, el  
n\'{u}mero de ranura en el bastidor y el n\'{u}mero de canal en la ranura, el tipo de  
m\'{o}dulo al que est\'{a} conectado el canal, el valor del voltaje, la tolerancia en este  
valor, y las fechas de creaci\'{o}n y de validez de la entrada en cuesti\'{o}n.\\

El siguiente paso fue la conexi\'{o}n de los cables de alto voltaje de C2 y C3 a los  
m\'{o}dulos de alto voltaje 1443PF, instalados, en el caso de C2 en el bastidor \#2 y en el  
de C3 en el bastidor \#5. Al encender los bastidores y elevar los voltajes se encontr\'{o}  
que fue necesario el reemplazo de las fuentes de poder del bastidor n\'{u}mero 2. Por  
otro lado se encontr\'{o} que dos tarjetas de alto voltaje 1443PF estaban da\~{n}adas,  
una en el bastidor \#2 y otra en el bastidor \# 5, incapaces de dar el voltaje necesario en  
todos los canales, por lo que fue necesario reemplazarlas.\\
Despu\'{e}s de instalado el sistema de alto voltaje, se volvi\'{o} a revisar las  
se\~{n}ales de cada canal, pero esta vez en el cuarto de control, observando las  
se\~{n}ales llegando al panel de interconexi\'{o}n. Se pudo observar que algunas  
se\~{n}ales no llegaban al panel de interconexi\'{o}n, a pesar de mostrar un buen  
funcionamiento cuando fueron revisadas a la salida del fototubo. Se encontr\'{o} que uno  
de los cables de se\~{n}al, correspondiente a C3-67, estaba incompleto, s\'{o}lo  
consist\'{\i}a de una secci\'{o}n de unos pocos metros desde el panel de  
interconexi\'{o}n necesit\'{a}ndose preparar y tender un nuevo cable de longitud  
adecuada para reemplazar este. Por otro lado, el problema con otra de las se\~{n}ales  
perdidas, la de C3-81, se debi\'{o} a que este cable no llegaba hasta el panel de  
interconexi\'{o}n, sin embargo en el conector 81 del panel hab\'{\i}a un cable conectado  
que result\'{o} ser el cable que originalmente estuvo conectado al fototubo 67, el que, en  
el momento que se descubri\'{o} esto, ya no estaba conectado por haber sido  
reemplazado. De modo que bast\'{o} con conectar al tubo 81 el cable inicialmente  
conectado al tubo 67.  
El control del alto voltaje en el E831 se realiza a trav\'{e}s de los programas HVOLT  
y COP, que interact\'{u}an con la base de datos y con las fuentes de alto voltaje.\\
\section{Revisi\'{o}n y activaci\'{o}n de los sistemas de gas} 
Una parte muy importante del proceso de implementaci\'{o}n de los detectores de  
Cerenkov es la referente al sistema de gases. Ya que los radiadores de estos detectores  
son gases, es necesario mantener estables las condiciones de estos estos gases para que no  
cambie su \'{\i}ndice de refracci\'{o}n, de otro modo los umbrales de los detectores  
ser\'{\i}an inestables y generar\'{\i}a confusi\'{o}n al momento de la identificaci\'{o}n  
de part\'{\i}culas.\\
Los detectores de Cerenkov del E831 trabajan, pr\'{a}cticamente, a presi\'{o}n  
atmosf\'{e}rica. La presi\'{o}n del gas es de alrededor de 0.1 pulgadas de agua por  
encima de la presi\'{o}n atmosf\'{e}rica para asegurar que no haya contaminaci\'{o}n  
por aire. Al haber una presi\'{o}n un poco mayor al interior del detector hay seguridad de  
que el aire del exterior no podr\'{a} ingresar. Pero, para poder mantener la presi\'{o}n  
deseada y mantener el flujo de gas en un nivel razonable fue necesario asegurarse de que  
no existan fugas de gas.  
Fue nitr\'{o}geno el gas con que se llenaron C2 y C3 en un primer momento para  
probar el sistema y verificar que no hubieran fugas. En el caso de C2 se encontr\'{o} que  
no pod\'{\i}a mantenerse la presi\'{o}n deseada, lo que indicaba la presencia de fugas de  
gas. El primer lugar para observar fugas de gas fue donde hab\'{\i}an sido reemplazados  
tubos fotomultiplicadores o de donde se hab\'{\i}an sacado para revisar los tubos o las  
bases y luego vuelto a instalar.  
Debido a que en C2 los tubos dan la cara directamente al volumen de gas radiador, al  
ser instalados desde afuera, la \'{u}nica manera de sellar la apertura por donde son  
introducidos es ajust\'{a}ndolos bien contra el borde de estas aperturas, las que tienen  
una cubierta de caucho para asegurar que se selle bien la zona. Despu\'{e}s de asegurar  
los tubos que fueron retirados el problema de la fuga de gas permanec\'{\i}a, de modo  
que fue necesario el uso de un detector electr\'{o}nico. Fue necesario retirar una cubierta  
de C2 para introducir una fuente de sonido, y por el exterior, con un detector que  
terminaba en un par de aud\'{\i}fonos se fue recorriendo todo el contador hasta encontrar  
los puntos por donde hab\'{\i}a escape de gas. Se encontr\'{o} que hab\'{\i}a fuga por  
las aperturas correspondientes a los tubos C2-61 y C2-91, que se detuvo con el ajuste de  
estos tubos contra los bordes de estas aperturas.\\

Despu\'{e}s de comprobar que la presi\'{o}n se manten\'{\i}a con un flujo de 5 pies  
c\'{u}bicos est\'{a}ndar por hora ( SCFH,  standard cubic feet per hour  ), se  
procedi\'{o} a llenar el contador con el gas usado como radiador Cerenkov, \'{o}xido  
nitroso N2O. Este gas se suple al contador mediante un sistema de botellas instaladas en  
el cuarto de gas norte, a trav\'{e}s de tuber\'{\i}as que llegan hasta el detector.  
El caso de C3 es un poco diferente al de C2. En este caso la presi\'{o}n del gas se  
pudo mantener sin problemas debido a que los fototubos no dan directamente la cara al  
volumen de gas sino que est\'{a}n separados de este por ventanas de cuarzo y de  
$CaF_{2}$ de modo que el gas est\'{a} confinado a una estructura interna bien sellada.  
Pero el sistema de gas de C3 consta de otra parte, el sistema de Nitr\'{o}geno para el  
drenaje del Helio que pueda difundirse a trav\'{e}s de las ventanas. El espacio entre las  
caras de los fototubos y las ventanas que dan al volumen de Helio en C3 est\'{a} lleno en  
todo momento de nitr\'{o}geno que, adem\'{a}s, est\'{a} en constante flujo y lavando  
cualquier cantidad de Helio que haya podido difundirse hasta aquel espacio. Sin este  
sistema, los fototubos acabar\'{\i}an contaminados con Helio, lo que provocar\'{\i}a  
mucho ruido y se\~{n}ales par\'{a}sitas debido a ionizaci\'{o}n de este gas.\\

La primera revisi\'{o}n de este sistema mostr\'{o} que se encontraba en mal estado  
ya que los borboteadores, que indican si hay flujo de nitr\'{o}geno por todos los tubos  
interconectados con uno de estos ( ver figura {\bf borb.ps } ), no mostraban ning\'{u}n  
burbujeo. La raz\'{o}n de esto era muy parecida al problema encontrado con C2, algunos  
tubos estaban mal ajustados contra los bordes de la apertura que deben sellar para que no  
haya escape de gas por ah\'{\i} y p\'{e}rdida de presi\'{o}n. Hubo que detectar  
cu\'{a}les eran los tubos mal ajustados conectando la manguera de salida de gas de cada  
tubo directamente al borboteador para observar si el recorrido de gas era el adecuado y no  
se escapaba por los bordes del tubo.  
El control del flujo de gases para C2 y C3 se realiza desde un panel de control al lado  
de cada detector, consta de v\'{a}lvulas de flujo y de medidores de presi\'{o}n  
Photohelic, los que controlan la apertura y clausura de las v\'{a}lvulas de entrada y salida  
de acuerdo al valor de la presi\'{o}n le\'{\i}da. La figura {\bf C3flowmts} muestra los  
controladores y medidores de C3, el flujo de Helio se fij\'{o} en 5 SCFH, valor en que se  
mantiene estable la presi\'{o}n, y el de Nitr\'{o}geno en 2SCFH, que permite un  
burbujeo estable en todos los borboteadores, lo que indica que hay el deseado flujo de  
este gas entre las ventanas de los fototubos y las ventanas que dan al volumen de Helio.  
El monitoreo de la presi\'{o}n del gas se realiza desde el cuarto de control donde se  
encuentra instalados medidores de presi\'{o}n para ambos detectores, adem\'{a}s de una  
alarma en caso de que la presi\'{o}n de Nitr\'{o}geno fluyendo entre los tubos de C3 baje  
de cierto l\'{\i}mite.\\
\section{Monitoreo de la Presi\'{o}n y la Temperatura} 
Fue mencionada ya la importancia de mantener en un estado estable los radiadores de  
los detectores de Cerenkov para que mantengan sus propiedades f\'{\i}sicas, en especial  
el \'{\i}ndice de refracci\'{o}n. Para poder observar cualquier cambio en estas  
condiciones ambos detectores tienen instalados medidores de presi\'{o}n y temperatura.  
La lectura de estas propiedades es realizada a trav\'{e}s de un volt\'{\i}metro digital  
modelo XXXXX que lee la se\~{n}al anal\'{o}gica del term\'{o}metro y del transductor  
de presi\'{o}n, convirti\'{e}ndolas a valores digitales que luego son traducidos a escalas  
de grados cent\'{\i}grados y pulgadas de agua de presi\'{o}n por el programa  
MONSTER, encargado en el E831 del monitoreo electr\'{o}nico del {\it hardware} del  
espectr\'{o}metro.\\
Para tener una historia de las condiciones de los radiadores de C2 y C3 durante todo  
el per\'{\i}odo de toma de datos se cre\'{o} un programa que cada cuatro horas,  
autom\'{a}ticamente, lee de los archivos temporales creados por MONSTER y escribe  
los \'{u}ltimos valores medidos de la presi\'{o}n y la temperatura de C2 y C3 en un  
archivo, C2C3TP.log. Un ejemplo de los datos guardados en este archivo puede  
observarse en el siguiente p\'{a}rrafo.\\
\begin{verbatim}  
 Date                      Ch_name         Std       Read    Error 
 May  1 04:26:37 C2TEMP-1    20.000   22.700    
 May  1 04:26:37 C2PRES-1   408.000  403.755    
 May  1 04:26:37 C3TEMP-1    20.000   17.250    
 May  1 04:26:37 C3PRES-1   408.000  400.894    
 
 Date                      Ch_name         Std       Read    Error 
 May  1 08:09:55 C2TEMP-1    20.000   21.450    
 May  1 08:09:55 C2PRES-1   408.000  406.701    
 May  1 08:09:55 C3TEMP-1    20.000   16.000    
 May  1 08:09:55 C3PRES-1   408.000  403.953    
\end{verbatim} 
\section{Instalaci\'{o}n del sistema ADC} 
 En un estante tipo FASTBUS ( FASTBUS crate ) se instalaron cuatro m\'{o}dulos  
convertidores anal\'{o}gico-digital ( ADC ) modelo LeCroy 1881M para C2 y C3. El  
convertidor 1881M de LeCroy contiene 64 canales de ADC con entradas integradoras de  
corriente, este m\'{o}dulo tiene como caracter\'{\i}sticas un tiempo de conversi\'{o}n  
corto, m\'{a}xima entrada de datos y un tiempo muerto corto. Los datos pueden ser  
comprimidos para reducir el volumen de datos y el tiempo de transmisi\'{o}n. El 1881M  
tiene una memoria capaz de almacenar hasta 64 eventos, esta memoria puede ser  
le\'{\i}da a un ritmo de 10 mega-palabras por segundo, de esta manera se reduce el  
tiempo muerto en el sistema de adquisici\'{o}n de datos. La lectura dispersadad de datos  
es  una opci\'{o}n en la que se puede restar del m\'{o}dulo datos no deseados, por  
ejemplo el fondo digitalizado cuando no hay presencia de pulsos durante la presencia de  
la puerta. Se pueden especificar 64 constantes, una para cada canal, que se pueden  
comparar con la mediciones. Solamente los datos que sobrepasan los valores de estos  
umbrales individuales son incluidos en la memoria de datos del evento. Esta  
caracter\'{\i}stica de lectura dispersa de datos puede reducir dram\'{a}ticamente la  
cantidad de datos que deben ser transmitidos a trav\'{e}s de FASTBUS para su  
procesamiento. Es en esta modalidad en que trabajan los ADC 1881M en el sistema de  
lectura de C2 y C3, el valor del ruido de fondo, llamado pedestal, es inclu\'{\i}do para  
cada canal y restado de las se\~{n}ales provenientes del detector.\\
 El sistema completo consiste, por una parte, de los cables de se\~{n}al provenientes  
de los \'{a}nodos de los fototubos, de las paletas conectoras entre el panel de  
interconexi\'{o}n, donde llegan los cables de se\~{n}al, y los ADC, y de los  
convertidores anal\'{o}gico-digital LeCroy 1881M. Por otro lado, se instal\'{o} el  
sistema de puerta maestra, puerta para los ADC, se\~{n}ales de permiso, borrado, y  
se\~{n}ales para el sistema de lectura de pedestales MONDA.  
 El funcionamiento del sistema es de la siguiente manera. Cuando los detectores que  
forman parte del sistema disparador detectan se\~{n}ales que indican que un evento con  
charm ha ocurrido estos env\'{\i}an una se\~{n}al llamada puerta maestra o {\it master  
gate}, esta se\~{n}al indica a los diferentes componentes del sistema de adquisici\'{o}n  
de datos que deben proceder a leer la informaci\'{o}n en los detectores porque es un  
evento que se desea almacenar. En el caso de los ADC de C2 y C3 esta se\~{n}al es  
llevada a un generador de puertas al que tambi\'{e}n llega la se\~{n}al de {\it spill} que es  
una se\~{n}al que dura todo el momento en que est\'{a} presente el haz de fotones  
interactuando con el blanco del experimento. Cada vez que hay coincidencia entre las dos  
se\~{n}ales el generador de puerta env\'{\i}a una al ADC y es durante el momento en  
que la puerta tiene un nivel diferente de cero que el ADC hace la integraci\'{o}n de la  
se\~{n}al que recibe en ese instante ( Ver figura \ref{mastgt.ps} ).\\

\begin{figure}[h]
\htmlimage{thumbnail=2.0}
\epsfxsize=3.0in
\centerline{\epsffile{mastgt.ps}}
\caption{Sistema de puerta para los ADC}
\label{mastgt.ps}
\end{figure}

 Una parte fundamental de la instalaci\'{o}n de este sistema fue la sincronizaci\'{o}n  
de las se\~{n}ales entrando en los ADCs y la puerta recibida por estos. Ya que el ADC  
realiza la integraci\'{o}n solamente durante el tiempo en que la puerta recibida est\'{a}  
activa, con valor l\'{o}gico 1, era necesario hacer que justo en ese momento fuera  
recibida la se\~{n}al del fototubo. Para esto se debi\'{o} hacer una medici\'{o}n del  
tiempo de llegada de la se\~{n}al para cada uno de los 210 canales respecto a la puerta  
maestra, la medici\'{o}n de estos tiempos permiti\'{o} establecer las longitudes precisas  
de los cables transmitiendo las se\~{n}ales de puerta para asegurar una integraci\'{o}n  
completa de los pulsos provenientes de los detectores. Se encontr\'{o} que los tiempos de  
llegada de las se\~{n}ales de uno de los detectores pod\'{\i}an dividirse en dos grupos.  
Para el caso de C2 hab\'{\i}a una diferencia de 40 ns entre los canales de fototubos de 5  
pulgadas de di\'{a}metro respecto a los canales de fototubos de 2 pulgadas de  
di\'{a}metro. Es por esto que para ese detector la conexi\'{o}n de los cables de  
se\~{n}ales a los ADCs tuvo que hacerse de manera que todos los canales de uno de los  
grupos estuvieran en un ADC y todos los canales del otro grupo en el otro ADC, cada uno  
de estos con un cable de diferente longitud proporcionando la puerta maestra al generador  
de puertas, el que a su vez env\'{\i}a la se\~{n}al de integraci\'{o}n a los ADCs.\\
   
\section{MONDA} 
 El sistema de monitoreo MONDA tiene como objetivo el control de los valores de los  
pedestales, que es la carga le\'{\i}da cuando no hay eventos. Esta carga �sin eventos� es  
originada por ruido presente en todo el sistema ( detectores, cables, fuentes de poder,  
m\'{o}dulos electr\'{o}nicos ), recogido del ambiente o por medio de la conecci\'{o}n de  
tierra. Este valor del pedestal para cada canal es el que se usa en los ADC con la  
opci\'{o}n de lectura dispersa de modo que se pueda descartar cualquier se\~{n}al que  
est\'{e} por debajo del umbral para cada canal.\\
 Para poner en uso MONDA para C2 y C3 se instal\'{o}, junto al sistema descrito en  
el punto anterior, un pulsador estrobosc\'{o}pico que env\'{\i}a durante el intervalo de  
tiempo en que no est\'{a} presente el haz de fotones ( {\it interspill} ) una serie de  
se\~{n}ales al generador de puertas. De esta manera se puede obtener una medida de la  
carga le\'{\i}da por los ADC en ausencia de eventos reales. Este valor de cada pedestal se  
resta a la lectura durante la presencia de haz para obtener s\'{o}lo la carga obtenida de  
fotoelectrones.\\
 El control de este sistema es efectuado por el programa c2ped ejecutado por  
�"mon\_sched", administrador de MONDA para todos los detectores con ADCs instalados  
en el mismo estante FASTBUS, desde el computador fc831 que controla el FSCC del  
estante donde se encuentran los ADC del sistema de Cerenkov junto con ADCs de otros  
detectores. C2ped se encarga de disparar el pulsador estrobosc\'{o}pico J41 que, a su vez,  
activa el generador de puerta m\'{o}dulo 2323 CAMAC si este recibe al mismo tiempo la  
se\~{n}al de {\it interspill}. Las puertas generadas por el m\'{o}dulo 2323 son recibidas  
por los ADC de C2 y C3 que integran la carga presente en cada canal durante la presencia  
de la puerta. La informaci\'{o}n le\'{\i}da durante cada {\it interspill} es escrita por c2ped  
en un archivo binario en el directorio /monda/cerenkov/c2/data con prefijo C2C3PED, la  
que posteriormente puede ser analizada para el estudio de los pedestales.\\
 El estudio de estos archivos mostr\'{o} caractr\'{\i}sticas importantes de los  
pedestales como el valor medio y la desviaci\'{o}n est\'{a}ndar. Muchos canales  
mostraban una desviaci\'{o}n est\'{a}ndar muy grande, lo que indicaba mucho ruido  
presente. En algunos casos era posible observar una distribuci\'{o}n con dos o tres picos  
distinguibles (fig ped anch-ang). Este ruido es muy indeseable debido a que causa una  
confusi\'{o}n grande en la distinci\'{o}n de se\~{n}ales reales. Por ello se hizo todo lo  
posible por reducir al m\'{a}ximo el ancho de la distribuci\'{o}n de se\~{n}ales de  
pedestal.  
 El primer lugar para limpiar de ruido el sistema eran los ADCs, sin conexiones  
externas algunas. La presencia de pedestales con valores de desviaci\'{o}n est\'{a}ndar  
mayores que 1.5 cuentas mostraban una falla en el ADC. Del mismo modo  
inestabilidades en el valor medio del pedestal a lo largo de un tiempo m\'{a}s largo, por  
ejemplo varias horas, mostraban tambi\'{e}n fallas en el ADC y, en consecuencia,  
requer\'{\i}an del cambio de dicho m\'{o}dulo.\\
 El segundo lugar para observar fuentes indeseables de ruido eran los cables y paletas  
conectoras, encargados de llevar las se\~{n}ales desde el panel de interconexi\'{o}n, a  
donde llegan los cables desde las bases de los fototubos, hacia los ADCs. No se  
encontr\'{o} en estos ninguna fuente extraordinaria de ruido, pero mediante la  
instalaci\'{o}n de capacitores de bloqueo en cada canal se redujo el ruido entrando al  
ADC.\\
 Se observ\'{o} que el ruido causante de la desviaci\'{o}n est\'{a}ndar tan grande en  
aquellos canales estaba presente desde la salida de la se\~{n}al desde las bases de los  
fototubos. Uno de los remedios a este ruido fue el aislamiento de los fototubos de 5� de  
di\'{a}metro (cubiertos con pintura met\'{a}lica) y de las bases conectadas a estos  
fototubos de la estructura de C3. El contacto de las bases y de los tubos con la estructura  
met\'{a}lica de C3 era responsable del recojo de ruido de 60 Hz, lo que hac\'{\i}a que la  
l\'{\i}nea de base oscilara hacia arriba y abajo respecto del nivel cero y que la carga del  
pedestal le\'{\i}da variase de la misma manera, creando una distribuci\'{o}n muy ancha  
el pedestal para estos canales. El aislamiento se logr\'{o} cubriendo con cinta adhesiva de  
aislante el\'{e}ctrico los fototubos, y con cinta de mylar las bases.\\
 Otra de las causas del pedestal excesivamente ancho en algunos canales fue el ruido  
introducido por la conexi\'{o}n de alto voltaje. En varios tubos de C3, sobre todo los de 5  
pulgadas, el ruido ten\'{\i}a una amplitud de hasta 4mV pico-pico, la frecuencia del ruido  
era de 62 kHz, causando que el pedestal para estos canales sea excesivamente ancho,  
incluso mostrando tres picos distinguibles ( ver figura X3 o 4? ). Se descubri\'{o} que el  
ruido proven\'{\i}a de la fuente de alto voltaje al comparar un canal con ruido y un canal  
sin ruido. Al momento de intercambiar cables de alto voltaje el ruido se trasladaba al otro  
tubo, y para asegurar que el ruido no estaba asociado al cable se intercambi\'{o} la salida  
de alto voltaje. Siempre el ruido estaba asociado a un canal espec\'{\i}fico de alto voltaje.  
 Para evitar este ruido se tuvo que reemplazar tres tarjetas de alto voltaje, pero en las  
nuevas tarjetas segu\'{\i}a presente el ruido en varios canales. Este problema no pudo ser  
resuelto por los expertos en el sistema y la \'{u}ltima medida para evitar el ruido  
excesivo fue, para estos canales, cambiar los cables a canales de alto voltaje sobrantes en  
la \'{u}ltima tarjeta. El cambio del pedestal con esta medida fue notorio, de un valor de  
10 cuentas para la desviaci\'{o}n est\'{a}ndar se redujo a un valor de 3.6.  
\section{Determinaci\'{o}n de los voltajes para los tubos fotomultiplicadores} 
 Ya que el valor del pedestal para cada canal de los ADC ser\'{a} sustra\'{\i}do para  
obtener s\'{o}lo la carga obtenida de fotoelectrones, se necesita que las se\~{n}ales reales  
no sean confundidas con \'{e}ste. Esto es, se desea que el pedestal est\'{e} separado lo  
suficiente de la se\~{n}al obtenida por un solo fotoelectr\'{o}n desprendido del  
fotoc\'{a}todo para no confundir una se\~{n}al real con pedestal y perderla. Para lograr  
esto se debe encontrar el valor apropiado del voltaje de alimentaci\'{o}n para cada tubo.  
 Uno de los medios para encontrar este valor de voltaje para cada fotomultiplicador fue  
el estudio de la respuesta de estos a pulsos de luz de diferente intensidad provenientes de  
diodos emisores de luz ( LED ). Para lograr esto se tuvo que montar paralelamente al  
sistema MONDA un arreglo que permitiera disparar los LEDs en el momento preciso que  
se enviara la se\~{n}al al generador de puertas y de esta manera hacer posible que los  
ADC integraran la se\~{n}al generada por los LEDs.\\
 El primer paso de este montaje fue la instalaci\'{o}n de LEDs en C2 y C3. En C2 se  
instalaron cuatro LEDs cerca a los bordes verticales, dos a cada lado, de la ventana  
anterior apuntando hacia los espejos verticales centrales los que reflejan la luz de los  
LEDs hacia los conos colectores de luz y los fototubos. En C3 se colocaron dos LEDs,  
uno cerca al borde inferior y otro cerca al borde superior de la ventana anterior apuntando  
hacia los espejos enfocadores, los que se encargan de reflejar la luz de los LEDs hacia los  
fototubos.\\
 La se\~{n}al que dispara los LEDs se origina en un m\'{o}dulo generador de pulsos  
estrobosc\'{o}picos J41, que es el mismo que env\'{\i}a las se\~{n}ales al generador de  
puertas usado en el sistema MONDA. Una de las salidas de este generador es enviada a  
un discriminador de m\'{u}ltiples salidas, una de estas es usada para disparar los LED y  
la otra para proporcionar la puerta al ADC encargado de integrar la se\~{n}al asociada al  
pulso de luz del LED. La figura led.ps es un diagrama de este sistema, pueden verse las  
diferentes etapas y m\'{o}dulos empleados para su instalaci\'{o}n y para la lectura de los  
pulsos obtenidos de los LED.\\
 La se\~{n}al proveniente del discriminador destinada para el disparo de los LEDs  
tiene que ser enviada primero a un convertidor NIM-TTL ya que el generador de pulsos  
necesita este tipo de se\~{n}al para ser activado. Recordemos que seg\'{u}n el protocolo  
NIM una se\~{n}al equivalente al 1 l\'{o}gico es de entre -0.7 y -1.6 V para una  
impedancia de 50 ohmios, mientras que para TTL el 1 l\'{o}gico es un pulso positivo de  
entre 2 y 5 V. Es el generador de pulsos el que se encarga de disparar los LEDs,  
pudi\'{e}ndose controlar la duraci\'{o}n y la amplitud de estos pulsos y, en  
consecuencia, controlar la intensidad de los pulsos de luz para obtener se\~{n}ales claras  
de los fototubos que nos den la informaci\'{o}n necesaria para determinar los voltajes  
apropiados para cada uno de estos.\\
 Adem\'{a}s de la instalaci\'{o}n del equipo arriba mencionado, se cre\'{o} el  
programa c2led y se instal\'{o} en el administrador MONDA de fc831. La funci\'{o}n de  
este programa es similar a la del programa c2ped, pero difiere de este en que activa el  
sistema de LEDs y lee la carga integrada por los ADC en respuesta a la luz captada por  
los tubos fotomultiplicadores escribi\'{e}ndola en archivos binarios con prefijo  
C2C3LED en el directorio /monda/cerenkov/c2/data.\\
 En la figura led.ps tambi\'{e}n se puede ver el sistema empleado para sincronizar las  
se\~{n}ales arrivando a los ADC con las puertas enviadas por el generador para que la  
integraci\'{o}n de la se\~{n}al sea precisa. Debido a que las se\~{n}ales provenientes del  
generador de pulsos hacia los LEDs tienen que viajar una distancia relativamente larga - 
desde el cuarto de conteo hasta los detectores en la planta baja- y que las se\~{n}ales  
enviadas por los fototubos tienen que recorrer una distancia similar hasta llegar a los  
ADCs, para que las puertas asociadas con los respectivos pulsos est\'{e}n en tiempo unas  
con otros, estas tienen que ser retrasadas el tiempo necesario. Para hallar este tiempo se  
us\'{o} un osciloscopio digital en donde se observaron ambas se\~{n}ales, la puerta  
entrando al ADC y la se\~{n}al del fototubo entrando tambi\'{e}n al ADC. Disparando el  
osciloscopio con la se\~{n}al de la puerta se pudo medir el tiempo de desfase y ajustar el  
valor del retardador para hacer coincidir ambas se\~{n}ales. Para poder observar en el  
osciloscopio la puerta enviada al ADC se tuvo que transformarla con un m\'{o}dulo  
ECL-NIM-ECL. El tiempo de conversi\'{o}n fue tambi\'{e}n tomado en cuenta al  
momento de hacer la sincronizaci\'{o}n.\\
 El resultado de las se\~{n}ales de luz proveniente de LEDs le\'{\i}das por los ADC  
para algunos canales se puede ver en la figura X2. En esta figura se puede observar la  
separaci\'{o}n entre el pedestal ( pico angosto y alto en el extremo izquierdo ) y la  
se\~{n}al de fotones ( distribuci\'{o}n m\'{a}s dispersa y de menor elevaci\'{o}n ).  
 La poca cantidad de LEDs colocados en los detectores caus\'{o} una distribuci\'{o}n  
desigual de la luz entre los fototubos. En una intensidad definida algunos canales no  
mostraban pedestal por recibir siempre luz con todas las puertas, en otros casos s\'{o}lo  
pod\'{\i}a observarse el pedestal porque muy poca luz llegaba a aquellos fototubos ( Ver  
la figura X3 ). Es por esto que se necesit\'{o} tomar diferentes conjuntos de datos  
variando la amplitud de los pulsos enviados a los LEDs para ver en todos los canales la  
distribuci\'{o}n de las se\~{n}ales de luz junto con el pedestal.  
 Otro de los m\'{e}todos empleados para la determinaci\'{o}n de los voltajes de  
alimentaci\'{o}n de los fototubos fue la medici\'{o}n, con el uso del osciloscopio digital,  
de la carga promedio de las se\~{n}ales por emisi\'{o}n t\'{e}rmica, comparables a las  
se\~{n}ales de un fotoelectr\'{o}n. Integrando en una ventana de la misma duraci\'{o}n  
efectiva empleada en los ADC, 60 ns, la carga presente cuando hay y cuando no hay  
se\~{n}al, pudimos ver a qu\'{e} voltaje la carga obtenida de estas se\~{n}ales era la  
apropiada para obtener una buena separaci\'{o}n entre estas y 
el pedestal.   
\section{Instalaci\'{o}n del sistema TDC} 
La instalaci\'{o}n del sistema TDC ( ver la figura X5 ) consisti\'{o}, para C3, en  
primer lugar, de la implementaci\'{o}n de dispositivos inversores de se\~{n}ales ya que  
las obtenidas de las bases, provenientes de un d\'{\i}nodo, eran pulsos positivos,  
necesit\'{a}ndose pulsos negativos para activar los amplificadores y luego los  
discriminadores. Uno de los intentos para obtener dos se\~{n}ales negativas del  
\'{a}nodo para ser usados en los ADCs y TDCs fue la instalaci\'{o}n de amplificadores  
en un estante tipo BIN colocado en la parte superior de C3. Ya que para cada entrada el  
amplificador da dos salidas, una de ellas pod\'{\i}a ser enviada al ADC mientras que la  
otra al TDC, pero se observ\'{o} que el efecto del amplificador sobre la se\~{n}al era tal  
que los pedestales para las se\~{n}ales amplificadas eran inadmisiblemente anchos, con  
una desviaci\'{o}\~{n} est\'{a}ndar de 50 cuentas o m\'{a}s. Se intent\'{o} reducir el  
ruido aislando el estante BIN de la estructura met\'{a}lica de C3, pero fue in\'{u}til, de  
modo que se tuvo que optar por tomar las se\~{n}ales de los d\'{\i}nodos e invertirlas  
antes de llevarlas a los amplificadores.  
Los inversores de se\~{n}ales fueron preparados usando filtros para baja frecuencia.  
Cada filtro consiste de un toroide de ferrita de 3.5 cm de di\'{a}metro exterior y 2 cm de  
di\'{a}metro interior enrollado con cable coaxial RG 174/U. El cable fue cortado y se  
colocaron dos capacitores de 47.4 mF conectados de modo que un extremo de cada  
capacitor estuviera conectado a la parte central del cable de uno de los extremos cortados  
y el otro estuviera conectado a la malla met\'{a}lica del otro extremo cortado. De esta  
manera se lograba la inversi\'{o}n de la se\~{n}al entrando al filtro modificado de esta  
manera. Se prepararon 110 inversores de se\~{n}ales, 100 de ellos se conectaron a las  
bases de los fototubos de C3 en la salida correspondiente al \'{u}ltimo d\'{\i}nodo.\\
 Otra de las tareas en este punto fue la medici\'{o}n e instalaci\'{o}n de los cables de  
se\~{n}ales que van conectados desde los inversores de se\~{n}ales, a su vez conectados  
a la salida de las bases de cada fototubo, hasta los amplificadores modelo LeCroy 612.  
Estos amplificadores se colocaron en un estante tipo NIM ( NIM bin ) que se instal\'{o}  
en una estructura modular ( rack ) situada en la planta baja, cerca de C3. Los m\'{o}dulos  
612 amplifican por un factor de diez las se\~{n}ales que recibe. Estas se\~{n}ales  
amplificadas son llevadas luego a discriminadores LeCroy 3412, que se instalaron en un  
estante CAMAC ( CAMAC crate ) instalado en el rack antes mencionado. El nivel de  
discriminaci\'{o}n se estableci\'{o} en 15 mV, el m\'{\i}nimo valor en estos  
discriminadores. Las se\~{n}ales discriminadas son luego transmitidas por cables  
entrelazados al segundo piso del edificio conectados a los TDCs, modelo LeCroy 3377,  
instalados en el rack 19, estante CAMAC 43 destinado para los detectores de Cerenkov.\\
El m\'{o}dulo 3377 es un convertidor tiempo-digital de 32 canales, las  
caracter\'{\i}sticas con las que es usado son las siguientes. Programado para tener el byte  
menos significativo (LSB) equivalente a 1 ns, el rango m\'{a}ximo es establecido en 160  
ns, la memoria de m\'{u}ltiples eventos fue programada para memorizar cuatro eventos,  
fue fijado para funcionar en modo de �parada com\'{u}n� ( common stop ), que significa  
que al llegar la se\~{n}al de puerta maestra es tomada como el momento de parar de  
incluir datos en la lectura del evento en cuesti\'{o}n. Esto es, en el momento de recibir la  
se\~{n}al de puerta maestra, los datos recibidos hasta 160 ns antes son le\'{\i}dos y  
transmitidos para la colecci\'{o}n de datos. La serie de m\'{o}dulos 3377 es le\'{\i}da en  
cadena v\'{\i}a el puerto ECL en cada uno de ellos, esto ofrece alta velocidad ( 100  
ns/palabra ) para la transmisi\'{o}n de datos.\\
En el caso de C2 este sistema estaba ya previamente instalado, aunque nunca fue  
utilizado, de modo que se tuvo que revisar todas las componentes del mismo. Se  
encontr\'{o} que la mayor\'{\i}a de los canales a la salida de los amplificadores no daban  
se\~{n}al alguna. El origen del problema en la mayor\'{\i}a de estos canales muertos se  
deb\'{\i}a a la falla de tres amplificadores que estaban completamente inoperantes,  
adem\'{a}s estos amplificadores originaban una falla en todo el estante NIM, de modo  
que adem\'{a}s de reemplazar los tres m\'{o}dulos en cuesti\'{o}n, tambi\'{e}n se  
reemplazo la fuente de poder del estante NIM con una fuente de mayor potencia.  
Despu\'{e}s de estos reemplazos s\'{o}lo quedaron unos cuantos canales muertos, pero  
se solucion\'{o} el problema trasladando los cables de aquellos canales al \'{u}ltimo  
amplificador en el estante que ofrec\'{\i}a canales no usados como reemplazo para  
aquellos da\~{n}ados en otros m\'{o}dulos.\\
Un problema similar al descrito arriba se present\'{o} en el caso de C3. En un  
momento desaparecieron las se\~{n}ales que recib\'{\i}amos de los d\'{\i}nodos, la causa  
de esto fue la falla del estante NIM, al usarse 9 m\'{o}dulos en el mismo estante la  
cantidad de corriente usada fue muy alta y la l\'{\i}nea de -12V fall\'{o}, dejando a los  
amplificadores sin poder. Fue necesario en este caso cambiar en los amplificadores el  
voltaje de funcionamiento para no sobrecargar la l\'{\i}nea de 12V. Posteriormente  
tambi\'{e}n fue reemplazada la fuente de poder del estante para evitar posibles fallas  
posteriores.\\
Despu\'{e}s de verificar que el conjunto del sistema estaba en funcionamiento  
qued\'{o} la tarea de reparar canales individuales para obtener se\~{n}ales claras. Fueron  
varias las causas de las fallas en los canales de TDC en ambos detectores. Algunos de los  
canales malos ten\'{\i}an la falla en el canal individual del amplificador en donde se  
hallaban conectados. Otra de las causas, en varios canales, fue un inversor de se\~{n}ales  
defectuoso. Otra causa se encontr\'{o} en las bases de un par de tubos, algunas  
conexiones se hallaban rotas y se hubieron de soldar.\\
Un problema serio en este sistema fue la presencia de canales ruidosos, lo que  
hac\'{\i}a imposible la distinci\'{o}n de las se\~{n}ales reales del fondo o ruido.  
Idealmente deb\'{\i}a obtenerse un pico central alto y a los lados picos peque\~{n}os, el  
pico central correspondiendo a se\~{n}ales en tiempo con la puerta maestra y los picos  
peque\~{n}os a se\~{n}ales en destiempo con la puerta maestra. En el caso de los canales  
ruidosos era imposible distinguir el pico central correspondiente a la se\~{n}al ( Ver  
figura X6? ). El origen de este ruido ven\'{\i}a principalmente de los amplificadores, que  
necesitaban terminadores de 50W en los canales no usados y en las salidas no usadas de  
cada canal. Pero algunos canales manten\'{\i}an el ruido y se decidi\'{o} cambiarlos a  
otro amplificador, se us\'{o} el \'{u}ltimo amplificador en el estante, con ocho canales  
disponibles. Esto redujo el ruido de estos canales, aunque a\'{u}n permanece un ruido  
notorio y alto comparado con otras se\~{n}ales de TDCs de estos detectores.\\
\section{Base de datos} 
Despu\'{e}s de la instalaci\'{o}n de los m\'{o}dulos y de su interconexi\'{o}n se  
procedi\'{o} a incluir en la base de datos del E-831 las direcciones de cada uno de los  
canales en el sistema de ADC�s y de TDC�s. La base de datos debe contener el nombre de  
cada canal de los detectores, 100 para C3 y 110 para C2. Junto con el nombre de cada  
canal se debe incluir todas las especificaciones necesarias que distingan  
inequ\'{\i}vocamente a cada canal, esto es, nombre del canal, n\'{u}mero de canal,  
n\'{u}mero de estante, ranura en el estante, subdirecci\'{o}n, y tambi\'{e}n algunos datos  
importantes como el valor del umbral a sustraerse al leer la carga en cada evento, el valor  
de la desviaci\'{o}n est\'{a}ndar del pedestal y el tiempo en que es valido el conjunto de  
datos para cada canal en la base de datos.\\
El programa usado en FOCUS para la creaci\'{o}n, mantenimiento e interacci\'{o}n  
con la base de datos es mSQL.\\
\section{Monitoreo del funcionamiento de los detectores} 
Despu\'{e}s de dejar los detectores en completo funcionamiento empez\'{o} la etapa  
en la que se debe verificar que este funcionamiento sea estable y la eficiencia permanezca  
alta. Para esto se usa el programa MONDA junto con un programa desarrollado  
especialmente para registrar cualquier variaci\'{o}n en el valor de los pedestales y  
actualizar estos valores en la base de datos y en el sistema de lectura FASTBUS. De esta  
manera se asegura la mejor calidad posible de los datos le\'{\i}dos y transmitidos a la  
corriente principal de datos.  
El programa para el monitoreo de los pedestales, llamado �ped�, lee los archivos  
binarios generados por MONDA, estos archivos tienen el valor promedio del pedestal y la  
desviaci\'{o}n est\'{a}ndar de su distribuci\'{o}n para cada {\it spill}, durante el cual se  
disparan 100 puertas. Estos valores le\'{\i}dos de MONDA son luego comparados con  
los valores de promedio y desviaci\'{o}n est\'{a}ndar en la tabla �fastbus� de la base de  
datos �focus� del E831. El umbral de cada canal, tambi\'{e}n inclu\'{\i}do en la base de  
datos, es establecido como el valor promedio del pedestal m\'{a}s cuatro veces el valor  
de la desviaci\'{o}n est\'{a}ndar. Si el valor medio le\'{\i}do por MONDA difiere en  
m\'{a}s de dos veces del valor de la desviaci\'{o}n est\'{a}ndar del valor medio que  
aparece en la base de datos o de la desviaci\'{o}n est\'{a}ndar del valor medio le\'{\i}do  
por MONDA, entonces un nuevo valor del umbral es introducido en la base de datos con  
valor igual al nuevo valor medio m\'{a}s cuatro veces la nueva desviaci\'{o}n  
est\'{a}ndar.\\
Si el cambio en el valor del pedestal es muy grande, esto es, mayor que cinco veces  
el valor de la desviaci\'{o}n est\'{a}ndar, una ventana aparece en la pantalla del terminal  
del cuarto de control dando opci\'{o}n de observar con mayor detenimiento el estado de  
los detectores y detectar alg\'{u}n posible problema que se haya presentado. Esta ventana  
da la posibilidad de realizar diferentes acciones, como mirar los valores relevantes de  
cada canal, actualizar la base de datos o pedir ayuda con instrucciones de c\'{o}mo  
proceder.\\
En el caso de producirse un cambio en el valor del pedestal y de actualizar la base de  
datos, para que el nuevo valor sea efectivo en el sistema de adquisici\'{o}n de datos se  
debe ejecutar un programa encargado de recompilar los programas del sistema FASTBUS  
y luego reinicializar el computador fc831 para que pueda ejecutar los nuevos programas.\\
El \'{u}ltimo paso en el proceso de monitoreo y actualizaci\'{o}n de pedestales tiene  
relaci\'{o}n con el an\'{a}lisis de los datos ya almacenados. Los programas de  
an\'{a}lisis de los datos de los contadores de Cerenkov necesitan saber el valor que tuvo  
el pedestal en cada canal para cada conjunto de datos que es analizado, esto es, para cada  
{\it run}. Un {\it run} es un segmento de ejecuci\'{o}n del experimento en que se toman  
datos suficientes para llenar una cinta magn\'{e}tica, o en el que las condiciones durante  
ese momento son especiales o distinguibles como para llevar un n\'{u}mero de  
identificaci\'{o}n propio.  
Se necesita tener un archivo en el que aparezcan los valores de los pedestales para  
cada canal y el n\'{u}mero de {\it run} a partir del cual tales valores se aplican. Esto  
quiere decir que cada vez que haya un cambio en alg\'{u}n pedestal, es necesario  
a\~{n}adir en este archivo el n\'{u}mero del canal, el valor del pedestal, y el n\'{u}mero  
de {\it run} a partir del cual este valor es v\'{a}lido.  
Para mantener este archivo actualizado se crearon algunos programas que en  
conjunto dan una lista de los nuevos cambios realizados a partir de la \'{u}ltima vez en  
que fueron ejecutados. El primer programa, /home/dolaya/dates0 cre\'{o} el archivo  
{\it run}sshort que contiene el n\'{u}mero de cada {\it run} junto con la fecha y la hora en  
que comenz\'{o} cada uno. Los pasos seguidos por este programa fueron la lectura del  
archivo donde se encuentran los datos referentes a cada {\it run} 
 
